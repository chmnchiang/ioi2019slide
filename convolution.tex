\documentclass[standalone]{beamer}

\begin{document}
\section{捲積}

\begin{frame}{\btitle{目的}}
  形如
  \[ c_k = \sum_{k = f(i, j)} a_i b_j \]
  的和,應用:
  \pause
  \disskip
  \begin{itemize}[<+->]
    \item 多項式乘法
    \item 大數乘法
    \item 組合計數 DP
  \end{itemize}

  \onslide<+->
  一般時間複雜度 $\ord(n^2)$。
\end{frame}

\begin{frame}{\btitle{多項式乘法}}
  令
  \begin{align*}
    \onslide<+->{f(x) &= a_0 + a_1 x + \dots + a_{n-1} x^{n-1},} \\
    \onslide<+->{g(x) &= b_0 + b_1 x + \dots + b_{n-1} x^{n-1},} \\
    \onslide<+->{f(x)g(x) &= c_0 + c_1 x + \dots + c_{2n-1} x^{2n-1}}
  \end{align*}
  \pause
  \[ c_k = \sum_{\color{red}{k = i+j}} a_i b_j \]
\end{frame}

\begin{frame}{\btitle{Karatsuba algorithm}}
假設 $f, g$ 皆為 $2n-1$ 次的多項式,也就是 
\[ f(x) \defeq a_0 + a_1 x + \dots + a_{2n-1}x^{2n-1} \]
\pause
令 
\begin{align*}
  f_1(x) &\defeq a_0 + a_1 x + \dots + a_{n-1} x^n \\
  f_2(x) &\defeq a_n + a_{n+1} x + \dots + a_{2n-1} x^n
\end{align*}
\pause
可以知道 $f(x) = f_1(x) + x^n f_2(x)$,類似的定義 $g_1, g_2$ 使得 $g(x) = g_1(x) + x^n g_2(x)$,
則
\begin{align*}
  f g &= f_1 g_1 + x^n (f_1 g_2 + f_2 g_1) + x^{2n} f_2 g_2 \\
  &= {\color<+(1)->{red}f_1 g_1} + x^n \big(
  {\color<.->{green!60!black}(f_1 + f_2)(g_1 + g_2)} - {\color<.->{red}f_1 g_1} - {\color<.->{blue}f_2 g_2}\big) 
  + x^{2n} {\color<.->{blue}f_2 g_2}
\end{align*}
\end{frame}

\begin{frame}{\btitle{Karatsuba algorithm}}
  只要算三個長度一半的多項式乘積。
  \pause

  複雜度:
  \[ T(n) = 3T(n/2) + \ord(n) \implies T(n) = \ord\left(n^{\log_2 3}\right) \]
  \pause
  實作上要跑的快常數小要注意:%
  \disskip
  \begin{itemize}
    \item 如果用 \cppinline{std::vector} 請用 \cppinline{std::vector::reserve}。
    \item 還是建議自己弄個 buffer。
  \end{itemize}
\end{frame}

\begin{frame}{\btitle{摸都嗨呀苦}}
  $n$ 個點 $(x, y)$ 決定了一個多項式,我們的夢想是這樣的:
  \pause

  \begin{enumerate}[<+->]
    \item \alert<6>{用 $(x_1, f(x_1)), (x_2, f(x_2)), \dots, (x_n, f(x_n))$ 表示 $f$。}
    \item 同樣的用 $(x_1, g(x_1)), (x_2, g(x_2)), \dots, (x_n, g(x_n))$ 表示 $g$。
    \item 那 $(x_1, f(x_1)g(x_1)), \dots, (x_n, f(x_n)g(x_n))$ 就表示 $f \cdot g$。\\
      \alert<.>{只要 $n$ 次乘法!}
    \item 再反推的 $f \cdot g$。
  \end{enumerate}
  \action<6-| alert@6>{光這一步就要 $\ord(n^2)$!}
  
  \onslide<7>{找的 $x_1, x_2, \dots, x_n$ 要夠好。}
\end{frame}

\def\myaddv{\vphantom{\omega_n^{()}}}
\begin{frame}{\btitle{離散傅立葉變換}}
  找 $x_k = \omega_n^k, \, 0 \leq k < n$,其中 $\omega_n \defeq \ex^{2 \pi \img / n}$。\\
  \pause
  現在 $f(x_k) = a_0 + a_1 \omega_n^k + a_2 \omega_n^{2k} + \dots + a_{n-1} \omega_n^{(n-1)k}
  \alert<2>{= \sum_{i=0}^{n-1} a_i \omega_n^{ik}}$。
  \pause

  也可寫成
  \[ 
    \begin{bmatrix}
      f(x_0) \\ f(x_1) \myaddv \\ f(x_2) \myaddv \\ \vdots \\ f(x_{n-1}) \myaddv
    \end{bmatrix}
    = 
  \begin{bmatrix}
    1 & 1 & 1 & \cdots & 1 \\
    1 & \omega_n^{1} & \omega_n^{2} & \cdots & \omega_n^{(n-1)} \\
    1 & \omega_n^{2} & \omega_n^{4} & \cdots & \omega_n^{(2n-2)} \\
    \vdots & \vdots & \vdots & \ddots & \vdots \\
    1 & \omega_n^{(n-1)} & \omega_n^{(2n-2)} & \cdots & \omega_n^{(n^2-1)} \\
  \end{bmatrix}
    \begin{bmatrix}
      a_0 \\ a_1 \myaddv \\ a_2 \myaddv \\ \vdots \\ a_n \myaddv
    \end{bmatrix}
  \]
  \pause
  把 $(f(x_0), \dots, f(x_{n-1}))$ 稱作 $A = (a_0, \dots, a_{n-1})$ 的\emph{離散傅立葉變換} $\mathcal{F}^n(A)$。
\end{frame}
\def\aat #1#2{\action<alert@#1>{#2}}
%\def\oclbb #1#2{\omega_{\textcolor{blue}{#1}}^{\textcolor{blue}{#2}}}
%\begin{frame}
  %如果 $n = 2m$
  %{
  %\[ 
  %\scalebox{0.85}{$\displaystyle
    %\begin{bmatrix}
      %\aat{2-3}{f(x_0)} \\ \aat{2-3}{f(x_1)} \myaddv \\ \aat{2-3}{f(x_2)} \myaddv \\ \vdots \\ \aat{2-3}{f(x_{m-1})} \myaddv \\ \vdots 
    %\end{bmatrix}
    %= 
  %\begin{bmatrix}
    %\aat{2-3}{1} & 1 & \aat{2-3}{1} & \cdots & \aat{2-3}{1} & 1 \\
    %1 & \omega_n^{1} & \omega_n^{2} & \cdots & \omega_n^{(n-2)} & \omega_n^{(n-1)} \\
    %\aat{2-3}{1} & \omega_n^{2} & \aat{2-3}{\alt<3->{\oclbb{m}{2}}{\omega_n^{4}}} & \cdots & \aat{2-3}{\omega_n^{(n-4)}} & \omega_n^{(2n-2)} \\
    %\vdots & \vdots & \vdots & \ddots & \vdots & \vdots \\
    %\aat{2-3}{1} & \omega_n^{(n-2)} & \aat{2-3}{\alt<3->{\oclbb{m}{1}}{\omega_n^{2(n-2)}}} & \cdots & \aat{2-3}{\omega_n^{(n-2)(n-2)}} & \omega_n^{(n-1)(n-2)} \\
    %\vdots & \vdots & \vdots & \ddots & \vdots & \vdots \\
  %\end{bmatrix}
  %{
    %\begin{bmatrix}
      %\aat{2-3}{a_0} \\ a_1 \myaddv \\ \aat{2-3}{a_2} \myaddv \\ \vdots \\ \aat{2-3}{a_{n-2}} \myaddv \\ a_{n-1} \myaddv
    %\end{bmatrix}
  %}
  %$}
  %\]
  %}
%\end{frame}

\begin{frame}[fragile]{\btitle{離散傅立葉變換}}
  \begin{python}[./genarr.py]
  \end{python}
  \[
    \action<2-| alert@2>{\mathcal{F}(A)_k}
    \action<3-| alert@3>{ = \mathcal{F}(\tikz[baseline, remember picture]{\node[overlay node] (_conv_E) {$E$}})_k}
    \action<5-| alert@5>{+\omega_n^{k} \mathcal{F}(\tikz[baseline, remember picture]{\node[overlay node] (_conv_O) {$O$}})_k}
  \]
  \begin{tikzpicture}[overlay, remember picture]
    \path<4->[->] ($(_conv_E.south)+(-0.4, -0.6)$) edge[bend right]
      ($(_conv_E.south)-(0, 0.1)$) node[left]{\ofootnotesize 偶數項} ;
    \path<3>[->,red] ($(_conv_E.south)+(-0.4, -0.6)$) edge[bend right]
      ($(_conv_E.south)-(0, 0.1)$) node[left]{\ofootnotesize 偶數項} ;
    \path<5->[->,red] ($(_conv_O.south)+(0.4, -0.6)$) edge[bend left]
      ($(_conv_O.south)-(0, 0.1)$) node[right]{\ofootnotesize 奇數項} ;
  \end{tikzpicture}
\end{frame}

\begin{frame}{\btitle{離散傅立葉變換}}
\begin{alignat*}{2}
  \onslide<+->{F(A)_k &= \sum_{j=0}^{n-1} \omega_n^{jk} A_j &&} \\
  \onslide<+->{&= \sum_{j=0}^{n-1} \ex^{2 \pi \img j k / n} A_j && \\}
  \onslide<+->{&= \sum_{j=0}^{m-1} \ex^{2 \pi \img \cdot (2j) \cdot k / n} A_{2j} &&\ + \  
  \sum_{j=0}^{m-1} \ex^{2 \pi \img \cdot (2j + 1) \cdot k / n} A_{2j+1}} \\
  \onslide<+->{&= \sum_{j=0}^{m-1} \ex^{2 \pi \img j k / m} E_{j} &&\ + \  
  \ex^{2 \pi \img k / n} \sum_{j=0}^{m-1} \ex^{2 \pi \img jk / m} O_{j}} \\
  \onslide<+->{&= \fourier(E)_{\alert<.->{k \bmod{m}}} &&\ + \ \ex^{2 \pi \img k / n} \fourier(O)_{\alert<.->{k \bmod{m}}}}
\end{alignat*}
\end{frame}

\begin{frame}{\btitle{離散傅立葉變換}}
  複雜度:
  \[ T(n) = 2 T(n/2) + \ord(n) \]
  \pause
  \[ \implies T(n) = \ord(n \log n) \]
\end{frame}

\begin{frame}{\btitle{離散傅立葉變換}}
  前面的「夢想」:
  \disskip
  \begin{enumerate}
    \setcounter{enumi}{2}
    \item 那 $(x_1, f(x_1)g(x_1)), \dots, (x_n, f(x_n)g(x_n))$ 就表示 $f \cdot g$。\\
      \alert<.>{只要 $n$ 次乘法!}
  \end{enumerate}
  \pause

  其怪? $f \cdot g$ 不是應該是 $2n+1$ 次多項式嗎?
  \pause

  我們其實是計算
  \[ c_k \defeq \sum_{\alert<.->{k \equiv i + j \operatorname{mod} n}} a_i b_j \]
  適當的在前面補 $0$ 還是可以計算多項式乘法!
\end{frame}

\begin{frame}{\btitle{離散傅立葉變換}}
  \begin{enumerate}
    \item \alt<2->{計算 $\fourier(A)$。\  ($\ord(n \log n)$ $\textcolor{mgreen}{\checkmark}$)}{
        用 $(x_1, f(x_1)), (x_2, f(x_2)), \dots, (x_n, f(x_n))$ 表示 $f$。}
    \item \alt<3->{計算 $\fourier(B)$。\  ($\ord(n \log n)$ $\textcolor{mgreen}{\checkmark}$)}{
        用 $(x_1, g(x_1)), (x_2, g(x_2)), \dots, (x_n, g(x_n))$ 表示 $g$。}
    \item \alt<4->{計算 $\fourier(C) = \fourier(A) \odot \fourier(B)$。\ ($\ord(n)$ $\textcolor{mgreen}{\checkmark}$)}{
        計算 $(x_1, f(x_1)g(x_1)), \dots, (x_n, f(x_n) g(x_n))$ 表示 $f \cdot g$。}
    \item \alt<5->{計算 $C = \fourier^{-1}(\fourier(C))$。\ \alt<8>{($\ord(n \log n)$ $\textcolor{mgreen}{\checkmark}$)}{\textcolor{red}{?}}}{反推 $f \cdot g$。}
  \end{enumerate}

  \begin{theorem}[離散傅立葉變換逆變換]<6->
    如果 $\fourier(A) = B$,則
    \[ A_k = \alert<6>{\frac{1}{n}} \sum_{j=0}^{n-1} \omega_n^{\alert<6>{- j k}} B_j \]
    %= \frac{1}{n}\sum_j \omega^{kj} B_j = \frac{1}{n} \sum_j (\overline\omega)^{-kj} B_j \]
    %其中 $\overline\omega_n = \ex^{- 2 \pi \im / n}$。
  \end{theorem}\disskip
  \onslide<7>{和正變換一樣,只要將 $\omega \to \omega^{-1}$}
\end{frame}

\begin{frame}[fragile]{\btitle{離散傅立葉變換}}
  和 Karatsuba algorithm 一樣,要實作上效率高需要巧思。

  \begin{minted}[autogobble, escapeinside=αα]{cpp}
void fft(int n, cplx a[], bool inv=false)
{
    int i = 0;
    for (int j = 1; j < n - 1; j++) {
        for (int k = n >> 1; k > (i ^= k); k >>= 1);
        if (j < i) swap(a[α\tikzmark{fft_i_start}αiα\tikzmark{fft_i_end}α], a[j]);
    }
    // ...
  \end{minted}
  \pause

  Question: 這個 \tikz[baseline, remember picture]{\node[overlay node] (fft_i_ques){\texttt{i}}} 依序是多少?
  \begin{tikzpicture}[remember picture, overlay]
    \node<2> [fill=cyan, fill opacity=0.2, circle, yshift=2,
    fit={(pic cs:fft_i_start) (pic cs:fft_i_end)}] {};
    \node (fft_i) [circle, yshift=2, fit={(pic cs:fft_i_start) (pic cs:fft_i_end)}] {};
    \path<2>[-latex, thick] (fft_i_ques.north) ++ (0, 0.1) edge[out=80,in=260] (fft_i);
  \end{tikzpicture}
\end{frame}

\begin{frame}[fragile]{\btitle{離散傅立葉變換}}
  \begin{minted}[autogobble, escapeinside=αα]{cpp}
double d = inv ? 1 : -1;
for (int m = 2; m <= n; m <<= 1) {
    int mh = m >> 1;
    for (int j = 0; j < mh; j++) {
        cplx w = exp(cplx(0, d * PI * j / mh));
        for (int k = j; k < n; k += m) {
            int l = k + mh;
            cplx x = a[k] - w*a[l];
            a[k] = a[k] + w*a[l];
            a[l] = x;
        }
    }
}
  \end{minted}
\end{frame}

\begin{frame}[fragile]{\btitle{另一個問題}}
  \begin{problem}[經典問題]
    有 $n$ 個技能 $m$ 個人,每個人都會這 $n$ 個技能中的某一些。如果 $B$ 會的所有技能 $A$
    也會,那我們就說 $A$ 完全贏過 $B$。對每個人輸出他完全贏過多少人(包含自己)。
    ($1 \leq n \leq 25$, $m \leq 10^6$)
  \end{problem}
  \pause \disskip
  你說這還不簡單: \disskip
  \begin{minted}{cpp}
for (int i = 0; i < (1<<n); i++) {
    for (int j: j | i == i) {
        dp[i] += cnt[j];
    }
}
\end{minted}
\pause \disskip
「你開心地想要直接揍他,一揍下去不得了痛死了,你像成龍一樣甩手時發現 \alert<.->{$n = 25$}。」
\footnote{引用自 2017 年 IOI-Camp 捲積講義}
\end{frame}

\begin{frame}[fragile]{\btitle{另一個問題}}
  \begin{minted}{cpp}
for (int k = 0; k < n; k++) {
    for (int i = 0; i < (1<<n); i++) {
        dp[k+1][i] = dp[k][i]; // 約定 dp[0][i] = cnt[i]
        if (i & (1 << k))
            dp[k+1][i] += dp[k][i ^ (1<<k)];
    }
}
\end{minted}
\pause \disskip
\cppinline{dp[k+1][i]} 表示「假設你會的技能的集合為 $i$,則你完全贏過,且你多會的技能只有在前 $k+1$ 個技能當中」的人數。
\pause
\disskip
\begin{enumerate}[<+->]
  \item 如果他\alert<.->{會}技能 $k$ 但你會 $\implies$ 他不會更前面的技能 \\
    $\implies$ 在 \cppinline{dp[k][i]} 算過。
  \item 如果他\alert<.->{不會}技能 $i$ 但你會 $\implies$ 在 \cppinline{dp[k-1][i]}算過
\end{enumerate}
\end{frame}

\begin{frame}{\btitle{另一個變換}}
  考慮多項式
  \[ F(x_1, \dots, x_n) = \sum_{I = \{i_1, \dots, i_k\} \subseteq [n]} a_I x_{i_1} x_{i_2} \cdots x_{i_k}
  = \sum_{I \subseteq [n]} a_I x_I \]
  $a_I$ 表示會的技能的集合為 $I$ 的人數。 \pause

  我們最後會得到一個新的多項式:
  \[ \tilde{F}(x_1, \dots, x_n) \defeq \sum_{I \subseteq [n]} \left( \sum_{J \subseteq I} a_J \right) x_I \]
\end{frame}

%\begin{frame}{\btitle{另一種看法}}
  %注意到
  %\[ E(x_1, \dots, x_n) = \sum_{I = \subseteq [n]} x_{i_1} x_{i_2} \cdots x_{i_k}
  %= (1 + x_1) (1 + x_2) \cdots (1 + x_n) \]
  %\pause
  %因此
  %\[ F(x_1, \dots, x_n) E(x_1, \dots, x_n) = F(x_1, \dots, x_n) (1+x_1)(1+x_2) \cdots (1+x_n) \]
%\end{frame}

\begin{frame}{\btitle{又一個捲積}}
  \begin{problem}[Two Swords Style, {\small Weekly Training Farm 16 pB}]
    給你 $a_0, a_1, \dots, a_{2^n-1}$,$b_0, b_1, \dots, b_{2^n-1}$,
    請你求出
    \[ c_k \defeq \sum_{k = i \texttt{|} j} a_i b_j \]
    對所有 $0 \leq k < 2^n$,其中 \texttt{|} 表示 bit OR。 ($n \leq 25$)
  \end{problem} \pause
  硬揍下去 $\ord(2^{2n}) \implies$ 痛的不得了。
\end{frame}

\begin{frame}{\btitle{做法}}
  \begin{enumerate}[<+->]
    \item 把每個數看成集合:如果 $x$ 在 $i_1, i_2, \dots, i_k$ 的 bit 為 $1$,則對應到
      $\{i_1, i_2, \dots, i_k\}$。
    \item 我們改成計算
      \[ \tilde{c}_x = \sum_{y \subseteq x} c_y \]
    \item 容易知道
      \[ \tilde{c}_x = \left(\sum_{y \subseteq x} a_y\right) \left(\sum_{y \subseteq x} b_y\right)\]
    \item 再用排容從 $\tilde{c}$ 得出 $c$。
  \end{enumerate}
\end{frame}

\begin{frame}{\btitle{另一種看法}}
  假設我們多項是的變數在 $\mathbb{F}_2$ 下,也就是 $x_i^2 = x_i$。令
    \[ F(\bm{x}) = \sum a_I \bm{x}_I,\ G(\bm{x}) = \sum a_I \bm{x}_I \]
  \pause

  我們要求的其實就是
  \[ F(\bm{x}) G(\bm{x}) = \sum_K \sum_{I \texttt{|} J = K} a_I b_J x_K \]
\end{frame}

\begin{frame}{\btitle{另一種看法}}
  \begin{enumerate}[<+->]
    \item 令
      \[ F(\bm{x}) = \sum a_I \bm{x}_I,\ G(\bm{x}) = \sum b_I \bm{x}_I \]
    \item 換成點值,$F(y)$ 表示「如果 $t \in y$ 則 $x_t = 1$,否則 $x_t = 0$」代入後的值。 
      \[ F(\bm{x}) \longleftrightarrow (F(0), F(1), \dots, F(2^n-1)) \]
    \item 計算點值相乘
      \[ (F(0)G(0), F(1)G(1), \dots, F(2^n-1) G(2^n-1)) \]
    \item 反推回去 $F(\bm{x})G(\bm{x})$。
  \end{enumerate}
\end{frame}

\begin{frame}{\btitle{另一種看法}}
  點值其實等於
  \[ F(x) = \sum_{y \subseteq x} a_y \]
  \pause
  有
  \[ \sum_{I \subseteq [n]} F(x) \tilde{\bm{x}}_I = \sum_{I \subseteq [n]} \left(\sum_{J \subseteq I} a_J\right) \tilde{\bm{x}}_I \]
  \pause
  這就是我們在「技能問題」中說的變換!
\end{frame}

\begin{frame}[fragile]{\btitle{另一種看法}}
  逆變換也很簡單:
  \begin{minted}{cpp}
for (int k = n-1; k >= 0; k--) {
for (int i = (1<<n)-1; i >= 0; i--) {
        dp[k][i] = dp[k+1][i];
        if (i & (1 << k))
            dp[k][i] -= dp[k][i ^ (1<<k)];
    }
}
\end{minted}
\end{frame}

\begin{frame}{\btitle{舉一反三}}
  \begin{problem}[經典問題]
    給你 $a_0, a_1, \dots, a_{2^n-1}$,$b_0, b_1, \dots, b_{2^n-1}$,
    請你求出
    \[ c_k \defeq \sum_{k = i \oplus j} a_i b_j \]
    對所有 $0 \leq k < 2^n$,其中 $\oplus$ 表示 bit \alert<.->{XOR}。 ($n \leq 25$)
  \end{problem}
\end{frame}

\begin{frame}{\btitle{舉一反三}}
  剛剛我們需要 $x^2 = x$,所以代值 $\{0, 1\}$。
  \pause

  現在我們需要 $x = x^{-1}$,代值 $\{-1, 1\}$。 \\
  也就是說我們改求點值
  \[ F(\bm{x}) \longleftrightarrow (F(0), F(1), \dots, F(2^n-1)) \] \pause
  其中 $F(y)$ 表示「如果 $t \in y$ 則 $x_t = \alert<.->{-1}$,否則 $x_t = \alert<.->{1}$」代入後的值。 
\end{frame}

\begin{frame}{\btitle{舉一反三}}
  令 $F(\bm{x}) = F_0(x_2, x_3, \dots, x_{n-1}) + F_1(x_2, x_3, \dots, x_{n-1}) x_1$,
  且 $I' = I \setminus \{1\}$ \pause
  \begin{alignat*}{2}
    \onslide<+->{\tilde{F}(\bm{x}) &= \sum_{I \subseteq [n]} F(I) x_I} && \\
    &\onslide<+->{= \sum_{1 \not\in I} F(I) x_I &&+ \sum_{1 \in I} F(I) x_I} \\
    &\onslide<+->{= \sum_{1 \not\in I} (F_0(I') + F_1(I')) x_I &&+ \sum_{1 \in I} (F_0(I') - F_1(I')) x_I} \\
    &\onslide<+->{= \sum_{1 \not\in I} (F_0 + F_1)(I') x_I &&+ \sum_{1 \in I} (F_0 - F_1)(I') x_I}
  \end{alignat*}
  \onslide<+->
  遞迴下去 $T(n) = 2T(n) + \ord(n) \implies T(n) = \ord(n \log n)$
\end{frame}

\begin{frame}{\btitle{習題}}
  這個變換又叫作 Walsh–Hadamard transform。 \pause
  \begin{exercise}
    \begin{enumerate}
      \item 找出 Walsh–Hadamard transform 的逆變換。
      \item 寫出 in-place 的 Walsh–Hadamard transform。
    \end{enumerate}
  \end{exercise}
\end{frame}
\end{document}
